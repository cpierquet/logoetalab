% !TeX TXS-program:compile = txs:///arara
% arara: pdflatex: {shell: yes, synctex: no, interaction: batchmode}
% arara: pdflatex: {shell: yes, synctex: no, interaction: batchmode} if found('log', '(undefined references|Please rerun|Rerun to get)')

\documentclass[french,11pt,a4paper]{article}
\usepackage[utf8]{inputenc}
\usepackage[T1]{fontenc}
%\usepackage{DejaVuSerif}
%\usepackage[scale=1.125]{inconsolata}
\usepackage{amssymb}
\usepackage{logoetalab}
\usepackage{enumitem}
\usepackage{tikz}
\usepackage{soul}
\usepackage{codehigh}
\usepackage{multicol}
\usepackage{tabularray}
\usepackage{fontawesome5}
\usepackage{fancyvrb}
\usepackage{fancyhdr}
\fancyhf{}
\renewcommand{\headrulewidth}{0pt}
%\rhead{\sffamily\small\affloetalab[Legende]}
\lfoot{\sffamily\small [logoetalab]}
\cfoot{\sffamily\small - \thepage{} -}
\rfoot{\hyperlink{matoc}{\small\faArrowAltCircleUp[regular]}}
\usepackage{hologo}
\providecommand\tikzlogo{Ti\textit{k}Z}
\providecommand\TeXLive{\TeX{}Live\xspace}
\providecommand\PSTricks{\textsf{PSTricks}\xspace}
\let\pstricks\PSTricks
\let\TikZ\tikzlogo

\usepackage{hyperref}
\urlstyle{same}
\hypersetup{pdfborder=0 0 0}
\usepackage[margin=2cm]{geometry}
\setlength{\parindent}{0pt}

\def\TPversion{0.1.0}
\def\TPdate{17 août 2023}
\def\HtRet{0.45}\def\LgRect{1.5}
\usepackage{tcolorbox}

\def\ListeTailleTexte{tiny,scriptsize,footnotesize,small,normalsize,large,large,LARGE,huge,Huge}

\sethlcolor{lightgray!25}
\NewDocumentCommand\MontreCode{ m }{%
	\hl{\vphantom{\texttt{pf}}\texttt{#1}}%
}

\usepackage{babel}

\begin{document}

\pagestyle{fancy}

\thispagestyle{empty}

\begin{center}
	\begin{minipage}{0.75\linewidth}
	\begin{tcolorbox}[colframe=yellow,colback=yellow!15]
		\begin{center}
			\begin{tabular}{c}
				{\Huge \texttt{etalab}}\\
				\\
				{\LARGE Logo de la licence \og Licence Ouverte Etalab 2.0 \fg.} \\
				\\
				\url{https://www.etalab.gouv.fr/licence-ouverte-open-licence/} \\
				\\
				{\small \texttt{Version \TPversion{} -- \TPdate}}
		\end{tabular}
		\end{center}
	\end{tcolorbox}
\end{minipage}
\end{center}

\begin{center}
	\begin{tabular}{c}
	\texttt{Cédric Pierquet}\\
	{\ttfamily c pierquet -- at -- outlook . fr}\\
	\texttt{\url{https://github.com/cpierquet/logoetalab}}
\end{tabular}
\end{center}

\hrule

\phantomsection

\hypertarget{matoc}{}

\tableofcontents

\vspace*{5mm}

\hrule

\vspace*{5mm}

\vfill

\hfill
{\Huge\sffamily\affloetalab[Legende]}
\hfill~

\bigskip

\hfill
{\Huge\loetalab*[Couleur=Bleu,Legende,TexteLegende={Licence Ouverte}]}
\hfill~

\vspace*{1cm}

\hfill
{\LARGE\loetalab~\loetalab[Couleur=Bleu]~\loetalab[Couleur=Gris]~\loetalab[Couleur=Marron]~\loetalab[Couleur=Rouge]~\loetalab[Couleur=Vert]~\loetalab[Couleur=Violet]}
\hfill~

\vspace*{1cm}

\hfill
{\LARGE\loetalab*~\loetalab*[Couleur=Bleu]~\loetalab*[Couleur=Gris]~\loetalab*[Couleur=Marron]~\loetalab*[Couleur=Rouge]~\loetalab*[Couleur=Vert]~\loetalab*[Couleur=Violet]}
\hfill~

\vfill~

\pagebreak

\section{Le package logoetalab}

\subsection{Idées}

L'idée est de pouvoir intégrer, dans un document \LaTeX, un logo relatif à la licence \og Licence Ouverte Etalab 2.0 \fg{} (\url{https://www.etalab.gouv.fr/licence-ouverte-open-licence/}).

\medskip

\hfill
\begin{tblr}{width=15cm,colspec={|[1pt]X[j]},cells={font=\footnotesize}}
	La « Licence Ouverte / Open License » présente les caractéristiques suivantes :\\
	1. Une grande liberté de réutilisation des informations :\\
	-- Une licence ouverte, libre et gratuite, qui apporte la sécurité juridique nécessaire aux producteurs et aux réutilisateurs des données publiques ;\\
	-- Une licence qui promeut la réutilisation la plus large en autorisant la reproduction, la redistribution, l’adaptation et l’exploitation commerciale des données ;\\
	-- Une licence qui s’inscrit dans un contexte international en étant compatible avec les standards des licences Open Data développées à l’étranger et notamment celles du gouvernement britannique (Open Government Licence) ainsi que les autres standards internationaux (ODC-BY, CC-BY 2.0).\\
	2. Une exigence forte de transparence de la donnée et de qualité des sources en rendant obligatoire la mention de la paternité.\\
	3. Une opportunité de mutualisation pour les autres données publiques en mettant en place un standard réutilisable par les collectivités territoriales qui souhaiteraient se lancer dans l’ouverture des données publiques.\\
\end{tblr}
\hfill~

\medskip

Les logos sont au format (vectoriel) \MontreCode{pdf}, et ont été obtenus à partir d'un fichier fichier \MontreCode{svg} sous licence CC BY 2.0 FR (\href{https://fr.m.wikipedia.org/wiki/Fichier:Logo-licence-ouverte2.svg}{[lien]}).

\subsection{Chargement}

Le package se charge dans le préambule, via \MontreCode{\textbackslash usepackage\{logoetalab\}}.

Les seuls packages chargés sont \MontreCode{graphicx}, \MontreCode{calc} et \MontreCode{simplekv}.

\begin{codehigh}[language=latex/latex2,style/main=cyan!10,style/code=cyan!10]
\documentclass{article}
...
\usepackage{logoetalab}
...
\end{codehigh}

\subsection{Utilisation}

Le package propose deux commandes :

\medskip

\begin{itemize}
	\item \MontreCode{\textbackslash loetalab} pour un affichage du logo en mode \textit{en ligne} (raccourci de \texttt{LicenceOuverteEtalab}) ;
	\item \MontreCode{\textbackslash affloetalab} pour un affichage \textit{autonome} (raccourci de \texttt{AfficheLicenceOuverteEtalab}).
\end{itemize}

\medskip

\begin{codehigh}[language=latex/latex2,style/main=cyan!10,style/code=cyan!10]
\loetalab(*)[Options]

\affloetalab(*)[Options]
\end{codehigh}

La différence entre les deux se situe au niveau de la taille et de l'alignement vertical du logo.

\begin{demohigh}[language=latex/latex3,style/main=cyan!10,style/code=cyan!10,style/demo=cyan!10]
\loetalab{} / \loetalab* / \affloetalab{} / \affloetalab*
\end{demohigh}

\pagebreak

\section{Les commandes}

\subsection{Affichage en mode \textit{en ligne}}

La commande dédiée à un affichage \textit{en ligne} est \MontreCode{\textbackslash loetalab}.

\medskip

\begin{itemize}[leftmargin=*]
	\item La version étoilée force l'affichage du logo en format paysage.
	\item L'argument optionnel, et entre \MontreCode{[...]} propose les clés suivantes :
	\begin{itemize}
		\item la clé \MontreCode{Couleur} pour choisir une couleur, parmi \MontreCode{Noir/Bleu/Gris/Marron/Rouge/Vert/Violet}, et vaut \MontreCode{Noir} par défaut ;
		
		\hfill{\footnotesize si une couleur non existante est choisie, c'est \MontreCode{Noir} qui sera utilisée}
		\item le booléen \MontreCode{Legende} pour afficher une légende à côté du logo, qui vaut \MontreCode{false} par défaut ;
		\item la clé \MontreCode{TexteLegende} pour paramétrer le texte de la légende et qui vaut \MontreCode{Publié sous licence Etalab 2.0} par défaut.
	\end{itemize}
\end{itemize}

\medskip

La hauteur du logo est calculée (et fixée) en fonction :

\begin{itemize}
	\item de 90\,\% de la hauteur globale, \underline{dans la police courante}, de la boîte formée des lettres \fbox{qB} ;
	\item décalée vers le bas d'un peu moins que la profondeur, \underline{dans la police courante}, de la boîte formée de la lettre \fbox{q}.
\end{itemize}

\medskip

\begin{demohigh}[language=latex/latex2,style/main=cyan!10,style/code=cyan!10,style/demo=cyan!10]
\foreach \Taille in \ListeTailleTexte
{%
    \csname\Taille\endcsname\texttt{\Taille} : Essai de logo \loetalab{} en ligne.\par
}
\end{demohigh}

\begin{demohigh}[language=latex/latex2,style/main=cyan!10,style/code=cyan!10,style/demo=cyan!10]
\foreach \Taille in \ListeTailleTexte
{%
	\csname\Taille\endcsname\texttt{\Taille} : {\sffamily Essai de logo \loetalab*{} en ligne.}\par
}
\end{demohigh}

\begin{demohigh}[language=latex/latex2,style/main=cyan!10,style/code=cyan!10,style/demo=cyan!10]
{\small\sffamily Publication sous \loetalab*[Couleur=Rouge,Legende, TexteLegende={Licence Etalab 2.0}].}\par
{\LARGE\sffamily Publication sous \loetalab[Couleur=Vert,Legende, TexteLegende={Licence Etalab 2.0}].}\par
\end{demohigh}

\subsection{Affichage en mode \textit{autonome}}

La commande dédiée à un affichage \textit{autonome} est \MontreCode{\textbackslash affloetalab}.

\medskip

\begin{itemize}[leftmargin=*]
	\item La version étoilée force l'affichage du logo en format paysage.
	\item L'argument optionnel, et entre \MontreCode{[...]} propose les clés suivantes :
	\begin{itemize}
		\item la clé \MontreCode{Couleur} pour choisir une couleur, parmi \MontreCode{Noir/Bleu/Gris/Marron/Rouge/Vert/Violet}, et vaut \MontreCode{Noir} par défaut ;
		
		\hfill{\footnotesize si une couleur non existante est choisie, c'est \MontreCode{Noir} qui sera utilisée}
		\item le booléen \MontreCode{Legende} pour afficher une légende à côté du logo, qui vaut \MontreCode{false} par défaut ;
		\item la clé \MontreCode{TexteLegende} pour paramétrer le texte de la légende et qui vaut \MontreCode{Publié sous licence Etalab 2.0} par défaut.
	\end{itemize}
\end{itemize}

\medskip

La hauteur du logo est calculée (et fixée) en fonction :

\begin{itemize}
	\item du double de la hauteur globale, \underline{dans la police courante}, de la boîte formée de la lettre \fbox{B} ;
	\item décalée vers le bas de la moitié de la hauteur, \underline{dans la police courante}, de la boîte \fbox{B}.
\end{itemize}

\begin{demohigh}[language=latex/latex2,style/main=cyan!10,style/code=cyan!10,style/demo=cyan!10]
\foreach \Taille in \ListeTailleTexte
{%
	\csname\Taille\endcsname\texttt{\Taille} : \affloetalab{} en mode autonome.\par
}
\end{demohigh}

\begin{demohigh}[language=latex/latex2,style/main=cyan!10,style/code=cyan!10,style/demo=cyan!10]
\foreach \Taille in \ListeTailleTexte
{%
	\csname\Taille\endcsname\texttt{\Taille} : \affloetalab*{} en mode autonome.\par
}
\end{demohigh}

\begin{demohigh}[language=latex/latex2,style/main=cyan!10,style/code=cyan!10,style/demo=cyan!10]
	\foreach \Taille in \ListeTailleTexte
	{%
		\csname\Taille\endcsname\texttt{\Taille} : \affloetalab*[Couleur=Violet] en mode autonome.\par
	}
\end{demohigh}

\begin{demohigh}[language=latex/latex2,style/main=cyan!10,style/code=cyan!10,style/demo=cyan!10]
{\small\sffamily \affloetalab[Couleur=Rouge,Legende].}\par
{\LARGE\sffamily \affloetalab[Couleur=Vert,Legende,TexteLegende={Licence Etalab 2.0}].}\par
\end{demohigh}

\vfill

\section{Historique}

\verb|v0.1.0|~:~~~~Version initiale

\vspace*{15mm}

\pagebreak

\end{document}